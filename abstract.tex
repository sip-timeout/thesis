\begin{abstract}
    The selection of an adequate set of users for different purposes is a crucial component in many modern applications, including crowdsourcing platforms, social networks and user-based recommender systems. To assist application developers in the intricate task of
    employing user selection modules, we present \qlang{}, a
    novel declarative query framework that supports flexible user
    selection. \qlang{} allows specifying selection criteria that are personalized, high-level and context dependent via a dedicated SPARQL extension.
    In particular, the extended query language has embedded constructs for capturing the properties and similarity of (relevant parts of) user profiles/data, through which the appropriate users for a given context can be effectively identified. \icde{To capture user data with rich semantics and account for wide-ranging applications we use an RDF-based data representation.
    We then develop a new, generic definition of a similarity metric for complex, semantically-rich user data,} along with efficient algorithms for computing similarity scores. Our experimental study on real-life data indicates the effectiveness and flexibility of our approach.
\end{abstract}
