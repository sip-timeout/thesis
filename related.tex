\section{Related Work}
\label{sec:related}

\subsection{Diversification}
\label{sec:div}
The diversification of retrieved items, not necessarily users, has been extensively studied in two prominent fields: search and recommender systems. We next overview the most relevant approaches in these fields.

\paragraph*{Diverse Search Results}
%\label{sec:diversesearch}
For years now, \emph{search results diversification} has been  studied in the field of information retrieval (IR)  ~\cite{servajean2013profile,AgrawalGHI09,carbonell1998MMR}. A study ~\cite{vee2008efficientdiv} has shown that a diverse set of search query results has the ability to enhance user satisfaction rate significantly. Search engines which avoid applying diversification techniques on results are more likely to return redundant top-ranked documents and thus fail to answer the often ambiguous ~\cite{anagon2005sampling} user query. Apart from solving query ambiguity, diversification is also used to negate over-personalization of search results~\cite{radlinskiD20006persondiv}.
Drosou et al ~\cite{drosou2010search} proposes the following classification of diversity definitions: \emph{content-based} which characterizes diversity as a $p$-dispersion problem, where the goal is to choose $p$ out of $n$ results in a way that the minimum distance between any pair of chosen points is maximized. \emph{Novelty-based} definitions categorize an item as diverse with respect to a set of past items if it contains novel information. Lastly, \emph{coverage-based} definitions aim to retrieve a set of documents that ``cover'' the many interpretations a certain user's query might have. Also see~\cite{zheng2012coverage} for a survey. In our research, we are inspired by coverage-based diversification, which is particularly suitable when considering selecting representatives from a user repository. In contract with IR methods, which mostly consider the coverage of (possibly overlapping) result categories, user data in our setting has unique characteristics such as many dimensions and the need to ``cover'' ranges of values for each dimension, which must be considered in diversification solutions (Section~\ref{sec:notion}).

\paragraph*{Diverse Recommender Systems}
The goal of recommender systems is to predict the rating a user would assign to a certain item, where items with a high prediction rating are considered most relevant to the client. Diversification in this context, similarly to IR, is used to ensure that the items recommended by the system are also sufficiently different and capture various needs of the client. It is possible to diversify by utilizing the item's semantic data and using some of the methods 
reviewed in the previous paragraph~\cite{servajean2013profile}, but in recent years a \emph{collaborative filtering} (CF) approach ~\cite{su2009cfsurvey}, that relies on the ratings of other users rather than semantic information , is often preferred. Boim et al.~\cite{boim2011priori} suggests using an algorithm based on a modified cover-tree data structure in order to produce diverse CF recommendations. In ~\cite{yu2009variety} the notion of \emph{explanation-based diversity} is presented: in this context, an explanation for the selection of an item may reflect the other similar items the user has rated in the past and/or the similar users which caused a certain item to be selected; then, it is suggested that selecting items with diverse explanations would manifest itself with a diversified recommendation set. In contrast, in our context there are no ready ``explanations'' that can be derived from the user selection process, and then used for aposteriori diversification; but rather the user selection process is inherently driven by coverage and diversification considerations. Moreover, to our knowledge, coverage-based approaches have not been considered in the context of recommender systems.  

\subsection{User Selection}
We next describe prominent contexts in which user selection is considered.

\paragraph*{Expert Finding}
\emph{Expert finding} refers to the act of selecting the fittest users (in terms of relevant skill or knowledge) to perform a  specific task within a large source population of expert users. The field has been a main focus of research within the IR community~\cite{tang2011expertisehetro,campbell2003expertemail}, particularly in the domain of \emph{social networks}~\cite{zhang2007expertsoc,bozzon2013choosing}. Correspondingly, we also propose a method to ensure selection of the most suitable users for our goal, which is to procure a diverse set of opinions. The main difference is that while expert finding focuses on the most capable users, we aim to ``cover'' the full spectrum of available user profiles (e.g., both low and high skill as reflected in the source population). A line of work in expert finding considers the selection of a team or a set of experts. We elaborate on this more relevant line in Section~\ref{sec:diverseUserPrelim} .

\paragraph*{Crowdsourcing}
\emph{Crowdsourcing} is a type of participative online activity in which multiple Web users of varying knowledge, heterogeneity, and number, undertake a task. Typically, such activities involve an overall goal which is achieved by the completion of many micro-tasks by Web users. For instance, the cleaning of a large knowledge base may be achieved by posting many micro-tasks of verifying concrete facts from the knowledge base.  %Freely speaking, crowd-sourcing is the habit of turning to a group of people to obtain needed information and services. Crowd-sourcing is a popular tool used by the average user to accomplish daily tasks such as navigation using the Waze application, travel using the AirBnB platform and KickStarter to start a new business. It is also extremely useful for more advanced users such as computer scientists which turn to Amazon Mechanical Turk to conduct large-scale experiments. The human factor in any crowd-sourcing scenario is crucial, leading researchers to develop several approaches for user selection in a crowd-sourcing environment [todo:ADDREF].
Various studies have considered the filtering of users who undertake a task, by different criteria. This includes, in particular, the assessment of crowd worker skill and filtering low-skill workers~\cite{ipeirotis2010quality}; the filtering of low trust or spammer users~\cite{raykar2012spam}; and the filtering of slow or inefficient users~\cite{haas2015CLAMShell}. These works are orthogonal to ours: we assume all the properties of a user are given, and this may include skill/trust/efficiency metrics derived by automatic tools as the ones mentioned here. 

We also mention here~\cite{amsterdamer2016december}, a declarative tool that allows customized user selection from a repository of user profiles and may be used in the context of crowdsourcing. Our flexible model for user profiles is inspired by this work; however, they do not consider the diversification of selected users.

%\scream{User selection in social networks?}


\subsection{Diverse User Selection}
\label{sec:diverseUserPrelim} 

\paragraph*{Team Formation}
In social networks, \emph{Team Formation} is the task of finding a group of people with the necessary skill-set to perform a given task~\cite{lappas2009finding}. Several articles focus on the importance of minimizing communication cost among team members~\cite{lappas2009finding,kargar2012eff}, which could have a negative effect on the diversity of team members. Maintaining a high level of diversity is important in order to increase creativity which is required to handle complex tasks~\cite{buccafurri2014driving}. Cohen and Yashinski~\cite{cohen2017Crowdiv}, propose an algorithm for team formation with a desired diversity constraint. Diversity is based on user personal properties, similar to our research. However, Cohen and Yashinski tries to make a user selection which distributes as similarly as possible to a pre-defined distribution over user properties, while in our research we do not assume such a target distribution.

\paragraph*{User Selection For Opinions}
%\label{sec:opinionrel}
In Section~\ref{sec:bg} we have presented the many aspects of diverse user selection. The present work is motivated by the procurement of diverse \emph{opinions}, and thus considers the \emph{full range} of scores assigned to any property, accounting for e.g.\ low and high ratings. This is in contrast with the coverage of document topics or expert finding. The recent work of~\cite{wu2015hear} is the most relevant to ours since it also studies diverse opinion procurement. However, they do not consider multi-dimensional data nor customization. %While our approach explicitly relies on a predefined set of properties for the grouping of users, other approaches may attempt to \emph{compute} the ``best'' groups using clustering methods (e.g.~\cite{boim2011diversification}). However, for such approaches, the explanation and refinement of the groups may be highly cumbersome to a client, and thus they are not practical for customization. Previous work has studied customizable user selection in different setting (e.g.,~\cite{amsterdamer2016december,fan2015icrowd}), and is complementary to our present study which focuses on diversification.